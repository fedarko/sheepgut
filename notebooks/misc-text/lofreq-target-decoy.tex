LoFreq called 22, 9,641, and 79 rare mutations in the CAMP, BACT1, and BACT2 genomes, respectively. At the frequency threshold $p = 2\%$, NaiveFreq called a similar number of rare $p$-mutations (17, 10,520, and 100 for CAMP, BACT1, and BACT2, respectively). It turns out that the sets of rare mutations identified by LoFreq and by NaiveFreq at $p = 2\%$ are similar: the numbers of overlapping rare mutations between these groups are 15, 8,033, and 43 for CAMP, BACT1, and BACT2. This suggests that, at least for this dataset, LoFreq primarily detected rare mutations with frequency of at least 2\%. Here, we describe an analysis of FDRs which suggests that there exist many more lower-frequency rare mutations.

Using LoFreq's calls, the numbers of rare mutations per megabase for each genome are 17, 4,477, and 28 for CAMP, BACT1, and BACT2, respectively. We can estimate the FDR of LoFreq's calls for the BACT2 genome (using the CAMP genome as a decoy) as $\frac{17}{28} \approx 60.61\%$, a very large FDR, indicating that either most identified mutations are false or that selection of the CAMP genome as the decoy results in a highly inflated estimate of the FDR. Although NaiveFreq's calls at the frequency threshold $p = 2\%$ resulted in a lower estimated FDR of $\frac{13}{36} \approx 37.00\%$ for the BACT2 genome, this is still a high FDR that raises concerns about downstream analyses such as phasing.

On the other hand, the estimated FDR of LoFreq's calls for the BACT1 genome (still using the CAMP genome as a decoy) is only $\frac{17}{4,477} \approx 0.38%$; NaiveFreq at the frequency threshold of $p = 2\\%$ has a slightly lower estimated FDR of $\frac{13}{4,885} \approx 0.27%$. Although both LoFreq and NaiveFreq at $p = 2\%$ result in the reliable identification of rare mutations with low FDR, we are still interested in extending the set of identified rare mutations while controlling the FDR. For example, lowering the frequency threshold of NaiveFreq to $p = 0.5\%$ results in the identification of 17,069 rare mutations in the BACT1 genome (an additional 6,549 rare mutations as compared to $p = 2\%$) with a higher but still relatively low FDR estimate of 2.44%.\endinput