LoFreq called 22, 9,641, and 79 rare mutations in the CAMP, BACT1, and BACT2 genomes, respectively. At the frequency threshold $p = 2\%$, NaiveFreq called a similar number of rare $p$-mutations (17, 10,520, and 100 for CAMP, BACT1, and BACT2, respectively). It turns out that the sets of rare mutations identified by LoFreq and by NaiveFreq at $p = 2\%$ are similar: the numbers of overlapping rare mutations between these groups are 15, 8,033, and 43 for CAMP, BACT1, and BACT2. This suggests that, at least for this dataset, LoFreq primarily detected rare mutations with frequency of at least 2\%. Here, we describe an analysis of FDRs which suggests that there exist many more lower-frequency rare mutations.

Using LoFreq's calls, the numbers of rare mutations per megabase for each genome are 17, 4,477, and 28 for CAMP, BACT1, and BACT2, respectively. We can evaluate the FDR of LoFreq's calls in the same way as earlier: using the BACT1 genome as a target and the CAMP genome as a decoy, we estimate the FDR for BACT1 as $\frac{17.06}{4,477.12} \approx 0.38%$. This is similar to the FDR of NaiveFreq at the frequency threshold of $p = 2\%$, which had an estimated FDR (using BACT1 as a target and CAMP as a decoy) of $\frac{13.19}{4,885.31} \approx 0.27%$. Although both LoFreq and NaiveFreq at $p = 2\%$ result in the reliable identification of rare mutations with low FDR, we are still interested in extending the set of identified rare mutations while controlling the FDR. For example, lowering the frequency threshold of NaiveFreq to $p = 0.5\%$ results in the identification of 17,069 rare mutations in the BACT1 genome (an additional 6,549 rare mutations as compared to $p = 2\%$) with a higher but still relatively low FDR (still using CAMP as a decoy) of 2.44%.\endinput