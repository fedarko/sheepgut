LoFreq called 57, 16,131, and 84 mutations in the CAMP, BACT1, and BACT2 genomes, respectively. At the frequency threshold $p = 2\%$, the na\"ive method called a similar number of $p$-mutations (52, 17,822, and 115 for CAMP, BACT1, and BACT2, respectively). It turns out that the sets of mutations identified by LoFreq and by the na\"ive method at $p = 2\%$ are roughly similar: the numbers of overlapping mutations between these groups are 50, 14,516, and 48 for CAMP, BACT1, and BACT2. This suggests that, at least for this dataset, LoFreq primarily detected mutations with frequency of at least 2\%. Below, we describe an analysis of FDRs which suggests that there exist many more lower-frequency mutations.

Using LoFreq's calls, the numbers of mutations per megabase for each genome are 44, 7,491, and 30 for CAMP, BACT1, and BACT2, respectively. We can evaluate the FDR of LoFreq's calls in the same way as earlier: using the BACT1 genome as a target and the CAMP genome as a decoy, we estimate the FDR for BACT1 as $\frac{44.21}{7,490.97} \approx 0.0059$. This is similar to the FDR of the na\"ive approach at the frequency threshold of $p = 2\%$, which had an estimated FDR (using BACT1 as a target and CAMP as a decoy) of $\frac{40.33}{8,276.24} \approx 0.0049$. Although both LoFreq and the na\"ive approach at $p = 2\%$ result in the reliable identification of mutations with low FDR, we are still interested in extending the set of identified rare mutations while controlling the FDR. For example, lowering the frequency threshold of the na\"ive approach to $p = 0.5\%$ results in the identification of 24,393 mutations in the BACT1 genome (an additional 6,571 mutations as compared to $p = 2\%$) with a slightly higher but still relatively low FDR (still using CAMP as a decoy) of 0.0194.\endinput