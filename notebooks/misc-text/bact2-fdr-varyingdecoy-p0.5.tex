Our initially described target-decoy approach estimated a very high FDR for the BACT2 MAG:
at $p = 0.5\%$, NaiveFreq identified 193 rare mutations per megabase in CAMP and 582 rare mutations per megabase in BACT2, resulting in an initial FDR estimate of $\frac{193}{582} \approx 33.2\%$.
%
Applying context-dependent target-decoy approaches decreases the estimated FDR of identified mutations in BACT2.
%
Using only the CP2 mutations in CAMP as a decoy genome (with 132 mutations per megabase), we can reduce our FDR estimate to $\frac{132}{582} \approx 22.6\%$.
%
Using all possible single-nucleotide nonsynonymous mutations in CAMP as a decoy genome (with 213 mutations per megabase), we can reduce this estimate further to $\frac{213}{582} \approx 36.6\%$.
%
Finally, using all possible single-nucleotide nonsense mutations in CAMP as a decoy genome (with 477 mutations per megabase), we obtain an estimate of $\frac{477}{582} \approx 82.0\%$.
\endinput