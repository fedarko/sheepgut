Our initially described target-decoy approach estimated a very high FDR for the BACT2 MAG:
at $p = 0.5\%$, NaiveFreq identified mutation rates of $6.4 \times 10^{-5}$ and $1.9 \times 10^{-4}$ for the CAMP and BACT2 genomes, resulting in an initial FDR estimate of $\frac{6.4 \times 10^{-5}}{1.9 \times 10^{-4}} \approx 33.2\%$.
%
Here we show how applying context-dependent target-decoy approaches changes the estimated FDR of identified mutations in BACT2.
%
Using only the CP2 mutations in CAMP as a decoy genome (with a mutation rate of $4.4 \times 10^{-5}$), we can reduce our FDR estimate to $\frac{4.4 \times 10^{-5}}{1.9 \times 10^{-4}} \approx 22.6\%$.
%
Using all possible nonsynonymous mutations in CAMP as a decoy genome (with a mutation rate of $4.8 \times 10^{-5}$), we obtain a slightly higher estimate of $\frac{4.8 \times 10^{-5}}{1.9 \times 10^{-4}} \approx 24.5\%$.
%
Finally, using all possible nonsense mutations in CAMP as a decoy genome (with a mutation rate of $7.7 \times 10^{-5}$), we obtain a much higher estimate of $\frac{7.7 \times 10^{-5}}{1.9 \times 10^{-4}} \approx 39.7\%$.
\endinput