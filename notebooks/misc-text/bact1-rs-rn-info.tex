For the BACT1 genome, for example
(where $N$ = 1,947,597),
there are $J$ = 1,250,228 possible synonymous and
$K$ = 4,592,563 possible nonsynonymous single-nucleotide mutations.
%
Using $p = 0.5\%$ as the threshold with which we na\"ively call codon mutations,
the BACT1 genome contains
$X$ = 12,954 synonymous and
$Y$ = 4,937 nonsynonymous single-nucleotide mutations.
%
At $p = 0.5\%$, then, the ratio of observed to possible synonymous single-nucleotide mutations
($R_S = X / J = \frac{12,954}{1,250,228}
\approx 0.0104$)
is much larger than the ratio of observed to possible nonsynonymous single-nucleotide mutations
($R_N = Y / K = \frac{4,937}{4,592,563}
\approx 0.0011$)
for BACT1.
\endinput